%% Sample essay

\subsection*{Vitaly Repin. Charles Baudelaire and Sigmund Freud on Art}
\addcontentsline{toc}{subsection}{Vitaly Repin. Charles Baudelaire and Sigmund Freud on Art}

\begin{multicols}{2}
Charles Baudelaire (1821--1867) and Sigmund Freud (1856--1939) had very different backgrounds and life experiences.
Baudelaire was an artist who was writing his beautiful verses and prose in Paris in the middle of the XIX century.
Freud was an Austrian neurologist who worked mostly in Vienna in the late XIX and first half of the XX century.
What were their views on art and how their past affected it? My essay addresses these questions.

In one of the works published in the interwar period Freud clearly expressed his understanding of art's role in human life~[1]:

\begin{quote}
	Life, as we find it, is too hard for us; it brings us too many pains, disappointments and impossible tasks. In order to bear it we cannot dispense with palliative measures.
\end{quote}

And he understood art as one of such measures~[1]:

\begin{quote}
	Satisfaction is obtained from illusions \dots\ At the head of these satisfactions through phantasy stands
	the enjoyment of works of art.
\end{quote}

Similar understanding of art could be found in Baudelaire poetry~[2]:

\begin{quote}
	One should always be drunk \dots\ Drunk with what? With wine, with poetry, or with virtue, as you please.
\end{quote}

Why? In order ``not to feel the horrible burden of Time weighing on your shoulders'', Baudelaire answers~[2].
Typical Freudian idea expressed by French poet when Freud himself was only of thirteen years old.

But Baudelaire did not stop here. As an artist he was very interested in understanding the nature of art.
He expressed his views on this subject in the famous essay ‘The Painter of Modern Life’. Baudelaire insists that
art has a dual nature~[3]:

\begin{quote}
	Beauty is always and inevitably of a double composition \dots\
	Beauty is made up of an eternal, invariable element \dots\ and of a relative, circumstantial element, which will be \dots\
	the age, its fashions, its morals, its emotions.
\end{quote}

He criticized neoclassical painters (e.g., Jean Ingres) because they ignored the second element of beauty.
What qualities should an artist have? Baudelaire was inspired by child's attitude to life~[3]:

\begin{quote}
	The child sees everything in a state if newness; he is always \textit{drunk} \dots\ genius is nothing more nor less than \textit{childhood recovered}
	at will --- a childhood now equipped for self-expression with manhood's capacities and a power of analysis.
\end{quote}

What is common and what is different between Freud's and Baudelaire's views on art?  Both of them see art as a way to
cope with suffering. Baudelaire seems to be quite enthusiastic about the power of art, Freud is very sceptical~[1]:

\begin{quote}
	The mild narcosis induced in us by art \dots\ is not strong enough to make us forget real misery.
\end{quote}

Looking back in time Freud was living helps to understand some of the reasons behind his scepticism. He saw
First World War (1914--1918), 1918 flu pandemic and the rising of Nazism in Germany. Last process affected
him personally - Freud's books were burned in Germany and Freud himself needed to leave Vienna and move to London in 1938
when he was more than eighty years old~[4]. Baudelaire did not have a chance to see these catastrophic events for European
culture. His contemporary and friend was Gustave Flaubert (1821--1880) who was disappointed in politics and saw art as
the only ``really real''~[5]. Baudelaire saw the birth of a new modern city --- Haussmann's renovation of Paris changed
the capital of France drastically and provided artists with new opportunities which modern life had to offer.

Baudelaire studied the nature of art much deeper than Freud. The reason seems to be obvious --- he was an artist and
was interested in obtaining profound understanding of art. Freud was a doctor and was focused on the practical issues ---
how art helps to survive in the tough world we live in. Both lines of thought heavily influenced contemporary
civilization. Freud's ideas are used by the modern art therapy. Baudelaire's --- by the modern theory of art.

\bigskip

\noindent\textbf{\large References}

\begin{enumerate}[{[1]}]
\item Sigmund Freud. \emph{Civilization and Its Discontents (1930).} Translated by James Strachey.
\item Charles Baudelaire. \emph{Paris Spleen (1869).} Translated by Louise Varèse.
\item Charles Baudelaire. \emph{The Painter of Modern Life (1863).} Translated by Jonathan Mayne.
\item Sigmund Freud Themes. \emph{Emigration.} Sigmund Freud Museum, Vienna.
\item Michael S. Roth. \emph{Modernism and Art for Art's Sake, part 4 of 4. Video lecture.}
\end{enumerate}
\end{multicols}

\newpage
