%% Sample essay

\subsection*{Anne Julienne. Rousseau under the lens of Kant’s Enlightenment}
\addcontentsline{toc}{subsection}{Anne Julienne. Rousseau under the lens of Kant’s Enlightenment}

\begin{multicols}{2}
Kant defined the Enlightenment~[1] as a process of moral development or maturation of humanity in which individuals
become less dependent on others and more self-reliant in establishing and expressing their own views on all aspects of life but
especially with regard to religious or spiritual matters. Such self-reliance required the moral virtues of ``resolution and
courage''.

The need for courage might refer back to the ushering in of the Age of Science following the hesitation of Copernicus to
publish (1543), the cruel martyrdom of Giordano Bruno (1600), and the humiliation of Galileo (1633). In Kant’s own case, a
Prussian censor disallowed the publication of his \emph{Religion within the Limits of Reason Alone}\ and Kant had to wait patiently for
the then King to be replaced so as to regain his freedom to publish [2].

However, courage can be needed in less obvious ways. It takes courage to write something original or controversial because
the reading public might mock or judge harshly, leaving the writer feeling humiliated and rejected. This kind of courage is
immediately evident in the opening preface of Rousseau’s ``Discourse on the Arts and Sciences''~[3] where he writes that he
could ``expect only universal censure'' over ``the position which [he had] dared to take''.

Is this courage enough to warrant judging Rousseau as a man of the Enlightenment? This is a vexed issue that continues to be
discussed. As Delaney~[4] puts it: ``there is dispute as to whether Rousseau’s thought is best characterized as ‘Enlightenment’
or ‘counter-Enlightenment’ ''. This contradiction or ambivalence in Rousseau is exemplified in the above discourse which is a
perfect example of Kantian self-reliance and courage but which also contains a never-ending diatribe against ``enlightened''
thinking. Delaney again: ``The work is perhaps the greatest example of Rousseau as a ‘counter-Enlightenment’ thinker.''

It is important to make distinctions here among different meanings of ``enlightened''. For Kant, becoming ``enlightened'' is
clearly yoked to virtue that comes in aid of individual moral and spiritual maturation. It is not about being clever or learned; it
is not about having knowledge of facts or scientific theories, or about being able to compose music that many will admire. All
of the latter is closer to the sense of ``enlightened'' that Rousseau uses in this discourse. He is attacking a certain sort of vanity
and arrogance that becomes associated with the arts and sciences. These are the vices of followers, not of leaders.

This is made abundantly clear in Rousseau’s high praise of the leading lights of the Age of Science: ``Bacon, Descartes, Newton
--- these tutors of the human race had no need of tutors themselves.'' Only similarly self-reliant geniuses ``who feel in
themselves the power to walk alone in those men's footsteps and to move beyond them'' should be permitted to ``devote
themselves to the study of the sciences and the arts''. Finally, ``it is the task of this small number of people to raise monuments
to the glory of the human mind''.

Rousseau is making a distinction here between a true and a pretentious enlightenment. In our own times, a distinction has
been made between ``scientific'' and ``scientistic'' views, between ``true'' and largely pretentious science. A similar distinction
can be made in the arts where pretension is such an easy trap for the budding artist and art appreciator.

There are certainly some non-Kantian understandings of ``Enlightenment figure'' to which Rousseau would not belong. In
particular, his was a ``discordant voice''~[5] in the context of the French Enlightenment with its emphasis on this-worldly
happiness and naturalistic science. However, Kant was strongly influenced by Rousseau in the development of his own
practical philosophy dealing with ethics and the moral order. He loved to read Rousseau, so much so that he notoriously
missed his daily walk after receiving a copy of \emph{Emile.} Kant was a leading genius of the enlightenment and in Rousseau, he
could readily recognise a fellow genius. He would not have stopped at describing Rousseau merely as ``an Enlightenment
figure'': he would have wanted to classify Rousseau further as a leader and a genius of the Enlightenment, in the same rank as
Bacon, Descartes, Newton.

\bigskip

\noindent\textbf{\large References}

\begin{enumerate}[{[1]}]
\item Immanuel Kant, \emph{“An Answer to the Question: What is Enlightenment?”,} World Public Library, 2008.
\item Wayne P. Pomerleau, \emph{“Immanuel Kant: Philosophy of Religion”,} Internet Encyclopedia of Philosophy, 2011.
\item Jean-Jacques Rousseau, \emph{``Discourse on the Arts and Sciences'',} University of Adelaide eBooks, 2012.
\item James J. Delaney, \emph{``Jean-Jacques Rousseau (1712--1778)'',} Internet Encyclopedia of Philosophy, 2005.
\item William Bristow, \emph{“Enlightenment”,} Stanford Encyclopedia of Philosophy, 2010.
\end{enumerate}
\end{multicols}
